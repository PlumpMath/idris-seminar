Let's move on and start talking about more interesting examples of
dependently typed programming in Idris, now that we've installed it.
We've seen that in a dependently typed programming language like Idris,
we can define a value-dependent data type like vectors, which unlike lists
allow us to be sure that we do not, for instance, mistakenly take the
head of empty list.

Now, let's look at another example of dependently typed functional programming
in Idris. We can define an operator that concatenates two vectors and certify
that the length of the resultant vector is the sum of the length of the
first vector plus the length of the vector:

\begin{lstlisting}
(<+>) : Vect n a -> Vect m a -> Vect (n + m) a
(<+>) Nil ys = ys
(<+>) (x :: xs) ys = x :: xs ++ ys
\end{lstlisting}

However, it is not enough to just certify, in the type signature of our concatenation operator,
that the length of our resultant vector is correct in order to ensure that our operator
for concatenating two vectors is correct. We want to be sure that we do not for instance,
produce a vector whose with the contents of the second vector prepended to the contents of
the first vector, rather than a vector with the contents of the second vector appended to the
contents of the first, when using our operator for concatenating two vectors together.

We can further increase our confidence that our concatenation operator is implemented
correctly by proving that it has the associative property: i.e. an incorrect implementation, such as
one that prepends the second vector to the first vector rather than appending it, would
not have this property. Though, even still our concatenation could be incorrect: for
example we could just implement a concatenation operator that takes two vectors and
produces a vector with an element repeated as many times as the sum of the lengths of the
two vectors. If we also prove that a vector concatenated with the empty vector is equal
to the first vector and the empty vector concatenated with a vector is equal to the second
vector, we can then be sure that our concatenation operator is implemented correctly.

It turns out that there are many other cases where we might want to define an associative
binary operator for data types, which takes two instances of some data type and produces another
instance of said data type. Data types where we've defined operators like this are called
semigroups: i.e. they are members of the semigroup type class. Also, a data type
is called a monoid if it is a semigroup and also has a neutral element: i.e. if
both its semigroup operator applied to one of its instance and the neutral element is equal
to the same instance and if its semigroup operator applied to the neutral element and one of
its instances is also equal to the same instance.
